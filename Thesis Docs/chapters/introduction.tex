\chapter*{Introduzione} % l'asterisco fa in modo che non compaia "Capitolo 1"
\addcontentsline{toc}{chapter}{Introduzione} % aggiunge il capitolo al TOC

Per diverso tempo Unity ha lavorato per migliorare le prestazioni del proprio motore di gioco, ma il problema principale non era del motore in sé, quanto più l'approccio con cui veniva scritta la logica delle applicazioni. Infatti, la programmazione orientata agli oggetti (con i classici GameObject e MonoBehaviour), si basa su un design poco performante, non ottimizzabile e decisamente non scalabile.
Dunque, negli ultimi anni gli sviluppatori Unity hanno iniziato una profonda ristrutturazione dell'architettura del loro motore di gioco, modificando quella che era un'architettura basata sugli oggetti, in una nuova, che prende il nome di Data-Oriented Technology Stack (DOTS).

L'obbiettivo di DOTS è riassumibile in \emph{performance by default}. L'idea su cui si basa è che il primo istinto quando si crea qualcosa su Unity dovrebbe essere una buona approssimazione di basso livello della giusta soluzione; e non qualcosa di distaccato su cui bisognerebbe ragionare e che per essere migliorata dev'essere ripensata, ricostruita o buttata via in parte appunto perché non funzionale o performante.
Con un approccio orientato agli oggetti questo non succede (e non può succedere) proprio perché esiste un livello di astrazione che distacca ciò che vediamo e usiamo, da ciò che si trova sotto (l'effettiva realizzazione), per semplificare e rendere tutto ``più simile alla realtà''. Al contrario, con un approccio orientato ai dati, ci si trova già a basso livello, e diventa quindi facile ottenere buone prestazioni e rendere il codice ottimizzabile di default.
In particolare, DOTS permette di sfruttare al meglio i processori multicore, massimizzando le prestazioni delle applicazioni multi-threaded e sfruttando in modo efficiente le cache della CPU. Inoltre, il nuovo approccio orientato ai dati rivoluziona anche il modo di pensare e scrivere il codice, il quale oltre ad avere i benefici delle performance, diventa anche più pulito e organizzato.

DOTS è ancora in fase di sviluppo, infatti le librerie sono soggette a modifiche e vengono aggiornate continuamente, ma è già possibile utilizzarle e accedere alla documentazione ufficiale. Attualmente include i seguenti package: Entities (preview), \Csh{} Job System, Burst Compiler, Unity Physics (preview), Unity NetCode (preview), DSPGraph (experimental), Unity Animation (experimental), DOTS Runtime (preview).


L'obbiettivo dell'elaborato è analizzare principalmente i package Entities e NetCode, che verranno utilizzati per sviluppare un semplice prototipo di videogioco multiplayer. Questo dovrà implementare delle meccaniche di base quali movimento del personaggio e raccolta di oggetti di scena. Il prototipo sarà realizzato interamente mediante l'utilizzo di DOTS, al fine di mostrare i vantaggi offerti da questa architettura.


Nel Capitolo~\ref{cap:evoluzione} forniremo un background sui game engine ed i principali modelli su cui si basano. Dunque, tratteremo il percorso evolutivo che ha portato Unity alla realizzazione dell'architettura DOTS. Nel Capitolo~\ref{cap:ecs} tratteremo il package Entities, che rappresenta le fondamenta di tutto il nuovo design orientato ai dati di Unity. Successivamente, il Capitolo~\ref{cap:netcode} aprirà con un'introduzione ai principali modelli di networking e ai problemi che è necessario risolvere. Dopodiché, mostreremo la soluzione adottata da Unity, tramite i package Transport e NetCode. Nel Capitolo~\ref{cap:prototipo} analizzeremo gli aspetti salienti dello sviluppo del prototipo, spiegando il funzionamento delle classi principali, mentre nel Capitolo~\ref{cap:risultati} svolgeremo un'analisi qualitativa dei principali package dell'architettura, mostrando i risultati sperimentali ottenuti. Infine, nell'ultimo capitolo, forniremo una visione d'insieme dell'elaborato, traendo le relative conclusioni e ragionando sui possibili sviluppi futuri dell'architettura.
